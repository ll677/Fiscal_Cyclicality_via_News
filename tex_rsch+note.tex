%%%
\documentclass[12pt]{article}
%%%%%%%%%%%%%%%%%%%%%%%%%%%%%%%%%%%%%%%%%%%%%%%%%%%%%%%%%%%%%%%%%%%%%%%%%%%%%%%%%%%%%%%%%%%%%%%%%%%%%%%%%%%%%%%%%%%%%%%%%%%%%%%%%%%%%%%%%%%%%%%%%%%%%%%%%%%%%%%%%%%%%%%%%%%%%%%%%%%%%%%%%%%%%%%%%%%%%%%%%%%%%%%%%%%%%%%%%%%%%%%%%%%%%%%%%%%%%%%%%%%%%%%%%%%%
\usepackage{setspace,graphicx,epstopdf,amsmath,amsfonts,amssymb,amsthm,versionPO}
\usepackage{marginnote,datetime,enumitem,rotating,fancyvrb} % removed subfigure
\usepackage{subfigure,threeparttable,booktabs,comment}% added subfigure
\usepackage{harvard,pdflscape}
%\usepackage{float}
\usepackage{curves}
\usepackage[applemac]{inputenc}
\usepackage{titlesec}
\usepackage{setspace}
\usepackage{makeidx}
\usepackage{eurosym}
\usepackage{sgame}
\usepackage{epsfig}%
\usepackage{epstopdf}
\usepackage{indentfirst}
\usepackage{endnotes}
\usepackage[labelfont=bf]{caption}
\usepackage{caption}
\usepackage{booktabs}
\usepackage{longtable}
\usepackage{rotating}
\usepackage{pdflscape}
\usepackage{lscape}
\usepackage{threeparttable}
\usepackage{booktabs}
\usepackage[hidelinks]{hyperref}
\usepackage[hyphenbreaks]{breakurl}
\usepackage{anyfontsize}
\usepackage{hyperref}
\usepackage{comment}
\usepackage{xcolor}
%\usepackage{graphicx,amsfonts,amsmath}
%\usepackage{array}

\hypersetup{
    colorlinks,
    linkcolor={blue!75!black},
    citecolor={blue!75!black},
    urlcolor={blue!75!black},
}
\usepackage{stackengine}

\setcounter{MaxMatrixCols}{10}
%TCIDATA{OutputFilter=Latex.dll}
%TCIDATA{Version=5.50.0.2960}
%TCIDATA{<META NAME="SaveForMode" CONTENT="1">}
%TCIDATA{BibliographyScheme=BibTeX}
%TCIDATA{LastRevised=Tuesday, September 02, 2014 17:28:08}
%TCIDATA{<META NAME="GraphicsSave" CONTENT="32">}

\newcommand{\tabitem}{~~\llap{\textbullet}~~}
\DeclareGraphicsRule{.tif}{png}{.png}{`convert #1 `dirname #1`/`basename #1 .tif`.png}
\usdate
\excludeversion{notes}
\includeversion{links}
\iflinks{}{\hypersetup{draft=true}}
\ifnotes{\usepackage[margin=1in,paperwidth=10in,right=2.5in]{geometry}
\usepackage[textwidth=1.4in,shadow,colorinlistoftodos]{todonotes}}
{\usepackage[margin=1in]{geometry}\usepackage[disable]{todonotes}}

\let\oldmarginpar\marginpar
\makeatletter\let\chapter\@undefined\makeatother
\newcommand{\clearRHS}{\clearpage\thispagestyle{empty}\cleardoublepage\thispagestyle{plain}}
\setcounter{tocdepth}{2}
\captionsetup{skip=0pt}
\setlength{\textwidth}{6.5in} \setlength{\oddsidemargin}{0in}
\setlength{\textheight}{8.4in} \setlength{\topmargin}{-.3in}
\renewcommand{\baselinestretch}{1.5}
\renewcommand{\arraystretch}{1.3}
%\setstretch{2}
\makeindex
%\input{tcilatex}

%%Graphics path in Overleaf
%\graphicspath{ {../Figures/} }
\graphicspath{{Figures/}}

% added by me
%\newcommand*\InputTable[1]{\input{../Tables/#1.tex}}
%% The line above modified for Overleaf:
\newcommand*\InputTable[1]{\input{{Tables/#1.tex}}}
%\usepackage{subcaption}

%% Comments by the three authors: Matteo, Marianna and John
\newif \ifshowexplanationsMatt
%\showexplanationsMatttrue
\showexplanationsMattfalse
\newcommand \explainMatt[1]{\ifshowexplanationsMatt \textcolor{red}{\ [MB: #1]} \fi}

\newif \ifshowexplanationsMarianna
%\showexplanationsMariannatrue
\showexplanationsMariannafalse
\newcommand \explainMarianna[1]{\ifshowexplanationsMarianna\textcolor{blue}{\ [MK: #1]} \fi}

\newif \ifshowexplanationsJohn
%\showexplanationsJohntrue
\showexplanationsJohnfalse
\newcommand \explainJohn[1]{\ifshowexplanationsJohn\textcolor{green}{\ [JM: #1]} \fi}

\newif \ifshowexplanationsMitchell
%\showexplanationsMitchelltrue
\showexplanationsMitchellfalse
\newcommand \explainMitchell[1]{\ifshowexplanationsMitchell\textcolor{orange}{\ [LL: #1]} \fi}

%% Citations
% The style below displays all authors in a 2+ author paper when the citation appears in the text for the first time, and later displays only the first name.
%\usepackage{natbib}
% Cite: Author(Year)
\newcommand{\cn}[1]{\citeasnoun{#1}}
% Cite: Author
\newcommand{\cna}[1]{\citename{#1}}
% Cite: Year
\newcommand{\cny}[1]{\citeyear*{#1}}
% Cite with possessive: author's (year)
\newcommand{\cnp}[1]{\cna{#1}'s (\cny{#1})}

\usepackage{floatrow}

\begin{document}

\title{{\LARGE Planning Ahead: Measuring Fiscal Cyclicality With News}}

\author{%
        \vspace{-0.5cm} {\large Louis Liu}\thanks{%
        \href{mailto:ll677@cornell.edu}{ll677@cornell.edu}} \\
}

%\author{\normalsize{Matteo Benetton\thanks{benetton@berkeley.edu}}\\
%\normalsize{Haas School of Business,} \\
%\normalsize{University of California at Berkeley}
%\and
%\normalsize{Marianna Kudlyak\thanks{marianna.kudlyak@sf.frb.org}}\\
%\normalsize{Federal Reserve Bank of San Francisco and CEPR}
%\and
%\normalsize{John Mondragon \thanks{john.mondragon@sf.frb.org}}\\
%\normalsize{Federal Reserve Bank of San Francisco}
%}

\vspace{-3mm} 
\date{\today}
\maketitle

\vspace{-8mm} 
%\sloppy
\begin{abstract}
%\hyphenpenalty=4000
\begin{singlespace}
%\noindent

I contribute to a literature on fiscal cyclicality by measuring ex-ante and contemporaneous budget responses to the business cycle separately. This is accomplished by  measuring the effects of output growth and prior ex-ante news on the fiscal balance using an  instrumental variables approach. I find that the response of the fiscal balance to ex-ante news varies depending on the news window used. Industrial economies appear to respond with robust  countercyclicality to news occurring between the October months of the preceding two years,  while middle-income developing economies seem to respond countercyclically to news  occurring between the April months of the preceding and current years. 

%Third, using variation in leverage constraints as an instrument for parents' equity extraction, we find that, around the constraints threshold, parents having access to home equity raises the probability of a child becoming a homeowner by five times.

\end{singlespace}
\end{abstract}

%\newpage
%\begin{singlespace}
%\tableofcontents
%\end{singlespace}
%\newpage

\clearpage

%%%%%%%%%%%%%%%%%%%%%%%%%%%%%%%%%%%%%%%%%%%%%%%%%%%%%%%%%%%%%%%%%%%%%%%%%%%%%%%%%%%%%%%%%%%%%%%%%%%%%%%%%%%%%%%%%%%%%%%%%%%%%%%%%%%%%%%
%%%%%%%%%%%%%%%%%%%%%%%%%%%%%%%%%%%%%%%%%%%%%%%%%%%%%%%%%%%%%%%%%%%%%%%%%%%%%%%%%%%%%%%%%%%%%%%%%%%%%%%%%%%%%%%%%%%%%%%%%%%%%%%%%%%%%%%
\section{Introduction \label{Introduction}}
%%%%%%%%%%%%%%%%%%%%%%%%%%%%%%%%%%%%%%%%%%%%%%%%%%%%%%%%%%%%%%%%%%%%%%%%%%%%%%%%%%%%%%%%%%%%%%%%%%%%%%%%%%%%%%%%%%%%%%%%%%%%%%%%%%%%%%%
%%%%%%%%%%%%%%%%%%%%%%%%%%%%%%%%%%%%%%%%%%%%%%%%%%%%%%%%%%%%%%%%%%%%%%%%%%%%%%%%%%%%%%%%%%%%%%%%%%%%%%%%%%%%%%%%%%%%%%%%%%%%%%%%%%%%%%%

A substantial literature exists concerning fiscal cyclicality, particularly with regards to its variation between industrial and developing countries. While earlier work like \cn{gavin1997fiscal} and \cn{kaminsky2004when} found that fiscal policy was often procyclical in developing countries, \cn{jaimovich2007procyclicality}, by addressing reverse-causality concerns using an instrumental variables approach, found that fiscal policy appears to be countercyclical across both industrial and developing economies.

\cn{jaimovich2007procyclicality}, \cn{gavin1997fiscal}, and \cn{kaminsky2004when} measure fiscal cyclicality as the effect of a business cycle indicator (e.g. output growth or output gap) on a  fiscal indicator (expenditures, taxes, or balance). A positive association between the business cycle indicator and the fiscal balance (revenues minus expenditures) is interpreted as evidence of countercyclical fiscal policy, while a negative association is interpreted as fiscal procyclicality and the absence of a strong association is interpreted as fiscal acyclicality. This approach, however, conflates deliberate legislative budget changes with adjustments  from countercyclical automatic stabilizers. These distinct components of fiscal policy can potentially have different cyclical trends. In a recession, for example, stabilizers will behave countercyclically with reductions in tax revenues and increases in outlays to social insurance programs. The government, however, may still choose to budget procyclically by cutting discretionary expenditures or increasing tax rates. This heterogeneity isn't discernible by measuring the response of a fiscal indicator to the contemporaneous business cycle, as one component will dominate or the two will negate each other. 

In this paper I propose a partial solution, separating the effects of forward-looking budgeting behavior from the fiscal balance's real time response to the business cycle. Reasoning that present news shocks about future output inform ex-ante government budget planning but not the future behavior of automatic stabilizers, I measure the response of the fiscal balance to not only contemporaneous output growth but also to prior news shocks regarding the given period.

\clearpage
%\nocite{*}

\bibliographystyle{aernobold}
\bibliography{references}



\end{document}
